
\section{Framework}

The general idea is to apply image analogies using stereoscopic data to generate new stereoscopic data.
We simply present the ideas here and implementation details are provided in the following section.

The general steps are:
\begin{enumerate}
	\item Select patches in stereo database that match our monocular image patches
	\item Transfer the corresponding stereoscopic data to synthesize the output frame from our input frame
	\item Process it on a scale pyramid to work at different level of details
\end{enumerate}

\begin{figure}
	\centering
	\TODO{have a nice $L \to R, L' \to R'$ figure here.}
	\caption{Stereo analogy principle: (i) we compute the Nearest patches in our database, (ii) we transfer stereoscopic data, (iii) and do it at multiple levels of detail}
\end{figure}

\subsection{Selecting patches}
The selection of pixels or patches from the database corresponds to finding the Nearest Neighbor Field (NNF) between the patches of our new image and those of our database.

Brute-force computation is not tractable given the quantity of data we are facing.
We chose to implement the PatchWeb algorithm~\cite{Barnes11} that extends Patch Match~\cite{Barnes09} to multiple exemplar NNF computation.

\subsection{Transfering the stereoscopic data}
Having a candidate patch in our database (with its corresponding right/left frame patch\footnote{Without loss of generality, we assume that our input corresponds to a left frame in our database.}), we propose different transfer strategies, namely transferring: the \emph{whole patch}, the \emph{patch difference}, or the \emph{patch disparity} as illustrated in Figure~\ref{fig:transfers}.

\begin{figure*}[ht!]
	\centering
	\begin{subfigure}{0.3\textwidth}
	\centering
		$L' \to R'=R$
		\caption{Transferring the whole patch: can only work in the case of a sufficiently large database that would contain the style of our input with a valid disparity.}
	\end{subfigure}\hfill
	\begin{subfigure}{0.3\textwidth}
	\centering
		$L' \to R'=L'+(R-L)$
		\caption{Transferring the patch difference: reduces dependency on the exact patch style (similarly, one could reconstruct the patch from the gradient information).}
	\end{subfigure}\hfill
	\begin{subfigure}{0.3\textwidth}
	\centering
		$L' \to R'=\textrm{warp}_{L\to R}(L')$
		\caption{Transferring the disparity: could suffer from occlusions where warping is ill-defined.}
	\end{subfigure}
	\caption{Three tentative strategies for stereoscopic data transfer}
	\label{fig:transfers}
\end{figure*}

\subsection{Pyramidal processing}
Since different objects and features appear at different scales, we apply the aforementioned selection and transfer both at multiple scales and independently (having found a patch at a given scale doesn't constrain the patch at the other scales).
The pyramid representation is a key component for texture transfer to work.

